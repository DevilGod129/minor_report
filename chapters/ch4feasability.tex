\section{Technical Feasibility}

 The project demonstrates strong technical feasibility by combining electronics, communica
tion, and information processing in a practical and achievable manner. The Arduino Mega
 microcontroller acts as the brain of the system, providing sufficient processing power and I/O
 pins to accommodate multiple sensor inputs, including MPU6050 sensors, tactile push buttons,
 and optional flex sensors. These electronic components assist in accurately detecting hand
 gestures and finger movements. For communication, the ESP8266 module is considered for
 wireless data transfer between the hardware system and the game engine, facilitating real-time
 interaction with low delay. Serial communication protocols (either wired or wireless) will be
 utilized to ensure smooth and continuous data exchange from the Arduino to the computer
 running the game.

  The information processing aspect involves translating raw sensor inputs into meaningful com
mands for the game. This includes gesture detection using both switch activations and motion
 readings from the MPU. All of this will be managed using standard Arduino programming,
 with open-source libraries that expedite development. The modular hardware setup allows for
 f
 lexible testing, updates, and part replacement if necessary. Our design supports iterative devel
opment, meaning we can test and improve as we build. Key strengths of the project include its
 adaptability, cost-effective components, and future scalability, making it a reliable and feasible
 system to implement with the tools and skills available to our team.

\section{Economic Feasibility}

 The proposed system demonstrates strong economic feasibility with a cost-effective approach
 to hardware development and implementation. Most of the required electronic components
 and sensors are readily available in the local market and within our college resources. The
 primary components, like Arduino Mega, ESP8266, MPU sensors, and connecting wires, are
 economically accessible, with relatively low procurement costs compared to specialized gam
ing interface systems. The use of PLA filament provides a budget-friendly prototyping solu
tion, allowing multiple design iterations without significant financial investment. Open-source
 software platforms like Arduino IDE and Unreal rendering Engine further reduce development
 expenses by eliminating expensive proprietary software licensing costs.

 The project’s modular design enables incremental development, meaning team members can
 progressively invest in components as needed, spreading out potential expenses. Potential cost savings are achieved through utilizing existing college laboratory equipment and leveraging
 team members’ existing technical skills, which minimizes additional training or external con
sultation expenses. The overall economic viability is enhanced by the project’s scalable na
ture, potential for future refinement, and the use of widely available, low-cost technological
 components. The minimal financial requirements make this project an economically attractive
 research and development initiative within the current institutional infrastructure.

