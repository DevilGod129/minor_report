\begin{itemize}
    \item \textbf{Autonomous Line Following and Navigation:} \\
    The system will accurately follow predefined paths using IR sensor arrays, with a line detection accuracy of over 95\% on high-contrast surfaces. The square DC geared motors, controlled by an Arduino Mega, will ensure smooth and stable movement across hospital or hostel corridors. The robot will maintain a consistent speed suitable for indoor navigation (approx. 0.3–0.5 m/s), with minimal deviation from the line path even at junctions and turns.

    \item \textbf{Scheduled Pill Dispensing with Patient Recognition:} \\
    The top-mounted camera module, connected to the Raspberry Pi, will perform real-time facial recognition of patients using trained models with an expected accuracy of over 90\% in well-lit environments. Based on patient identity and the stored schedule, the system will trigger the appropriate servo motor to dispense the correct dosage from the designated pill compartment. The dispensing process will be completed within 3–5 seconds of successful identification.

    \item \textbf{Modular Operation and System Feedback:} \\
    Each compartment of the bot will operate semi-independently under a unified control system. The Raspberry Pi and Arduino will communicate via serial protocol to synchronize motion with dispensing operations. The system will provide LED or sound-based feedback for key actions—such as arrival at a patient’s bed, successful identification, and pill dispensing—ensuring operational transparency.

    \item \textbf{Real-Time Processing and Minimal Latency:} \\
    The facial recognition and pill dispensing processes will maintain a latency of under 1 second, ensuring quick and reliable patient interactions. The line-following subsystem will update motor commands at over 50 Hz, providing responsive and accurate path tracking. Power-efficient components will enable up to 1–2 hours of autonomous operation on a single battery charge, depending on the load.

    \item \textbf{Scalable and Adaptable Framework:} \\
    The modular design allows for easy expansion—additional compartments or storage trays can be added for more patients or different types of medicines. The robot can also be reprogrammed to follow new routes or modified to integrate RFID or QR code-based patient recognition as future enhancements.
\end{itemize}

\newpage

\begin{itemize}
    \setcounter{enumi}{5} % continue numbering if using enumerate, can be removed if unordered
    \item \textbf{Patient Database Management:} \\
    The system will maintain a secure and up-to-date patient database storing personal information, medication schedules, and dispensing history. This database will enable accurate patient identification, scheduling, and audit trails, and it can be accessed or updated remotely through a cloud-based interface for ease of management by healthcare staff.
\end{itemize}
