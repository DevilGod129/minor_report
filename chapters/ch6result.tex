\section{Expected System Functionality and Outcomes}

\subsection{Autonomous Line Following and Indoor Navigation}
The robot is designed to navigate indoor environments smoothly and independently using a line-following mechanism. An array of IR sensors continuously detects the predefined path laid on the floor, allowing the system to accurately follow routes without manual intervention. This ensures reliable navigation through structured indoor settings such as hospital corridors, hostels, or care facilities.  

The navigation system enables stable motion at a controlled speed suitable for indoor environments, reducing the risk of collisions and ensuring safe operation around patients and staff. The autonomous navigation capability minimizes the need for human supervision and allows the robot to perform repeated delivery tasks efficiently.

\subsection{Patient Recognition and Scheduled Medicine Delivery}
The system incorporates a patient recognition module to ensure that medicines are delivered to the correct individual at the right time. Using a camera-based recognition system, the robot identifies patients upon reaching their designated location. Once the patient is successfully recognized, the system verifies the corresponding medication schedule stored in its database.  

Based on this schedule, the robot dispenses the prescribed medicine automatically using a servo-controlled dispensing mechanism. This process ensures timely medication delivery, reduces human error, and improves adherence to prescribed treatment plans.

\subsection{Coordinated System Operation and Feedback Mechanism}
All subsystems of the robot operate in a coordinated manner under a unified control framework. The navigation, recognition, and dispensing modules work together seamlessly to complete the delivery task. Simple feedback mechanisms such as audio alerts are used to indicate important system states, including arrival at a patient location, successful patient recognition, and completion of medicine dispensing.  

This feedback improves transparency and allows caregivers or staff to easily monitor the robot’s operation without requiring technical expertise.

\subsection{Software Control, Scheduling, and Data Logging}
The robot’s software is responsible for managing medication schedules, patient data, and system operation. It maintains records of medicine dispensing events, including time, patient identity, and delivery status. These logs provide an audit trail that can be reviewed for monitoring and analysis purposes.  

By automating scheduling and record keeping, the system reduces administrative workload and helps ensure accurate and consistent medication management.
