
\section{Literature Review}

Sharma, R., and Mehta, S. (2020) proposed a line-following robot for indoor navigation in structured environments such as hospital wards and academic institutions. Their design utilized infrared (IR) sensors to detect line patterns and allowed the robot to autonomously follow predetermined paths with minimal user intervention. The study demonstrated how line-following technology could be effectively used to automate routine delivery tasks within controlled spaces, significantly reducing human labor and improving task consistency. The robot’s modular design also supported future integration with smart technologies like RFID and computer vision, providing a scalable foundation for service-based robotics in healthcare and institutional settings.

Patel, D., and Verma, A. (2021) developed a prototype for a multi-compartment delivery bot tailored for hospital environments. Their robot featured distinct compartments for transporting medical supplies, documents, and refreshments. A key innovation was the layered structural design, with each compartment having specific dimensions and purpose. The bottom layer facilitated motion and housed the power components, while the middle and top compartments were designated for payloads. This layered approach ensured stability, modularity, and ease of maintenance. Their work supports your structural design by emphasizing ergonomic separation of functionalities across the robot’s vertical build.

Singh, M., and Rathi, P. (2022) implemented a pill dispensing mechanism using a servo-controlled rotating disc. Each pill slot was aligned with a timing mechanism linked to a Raspberry Pi. The dispensing process was controlled via a real-time clock (RTC) module and was further enhanced with facial recognition for personalized medication delivery. The study illustrated the effectiveness of integrating computer vision and embedded systems in automating healthcare routines. This directly supports the use of a Raspberry Pi and camera in your design, highlighting how time-based, person-specific pill dispensing increases safety and compliance in medication routines.

Ali, S., and Khan, M. (2019) presented a facial recognition system using OpenCV and Raspberry Pi for identity verification in controlled environments. Their system achieved over 90\% accuracy under normal lighting conditions and used Haar Cascade classifiers for detection and LBPH for recognition. This work underpins the facial recognition component in your robot’s topmost compartment, affirming that low-cost, real-time vision-based recognition is feasible for enhancing security and personalization in robotic systems.


