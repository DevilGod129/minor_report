
\section{Literature Review}

Case Study: A Line Following Robot for Hospital Management
Authors: P. Kumaresan, G. Priya, B. R. Kavitha, G. Ramya, M. LawanyaShri
In their work Case Study: A Line Following Robot for Hospital Management, Kumaresan et al. present the design and development of an autonomous line-following robot intended to support hospital operations by transporting items such as medicines, linens, and equipment along predefined ground paths. The system uses infrared sensors to detect and follow floor lines, and an Arduino Uno microcontroller to process sensor data and navigate accordingly, demonstrating a cost-effective approach to automation in healthcare environments. A Bluetooth-controlled Android interface was included to allow remote interaction with the robot, enhancing its practical usability. This study highlights how simple sensor-based mobile platforms can reduce staff workload, improve service efficiency, and offer scalable assistance for routine logistics tasks within clinical settings. 

MedBuddy: The Medicine Delivery Robot
Authors: Akshet Patel, Pranav Sharma, Princy Randhawa
The MedBuddy: The Medicine Delivery Robot study by Patel, Sharma, and Randhawa explores a Bluetooth-controlled robotic system designed to deliver medicines within hospital wards while minimizing direct contact between healthcare workers and patients—particularly relevant during infectious disease outbreaks like COVID-19. Built on an Arduino Uno platform, MedBuddy integrates an ultrasonic sensor for dynamic distance detection and utilizes a smartphone’s camera to provide a live feed to a custom mobile application for remote teleoperation. The robot’s design addresses both navigation and safety concerns and demonstrates how accessible electronics and wireless communication can be combined to create a practical medicine delivery solution that reduces the risk of infection and supports safe, timely distribution of medications.
Singh, M., and Rathi, P. (2022) implemented a pill dispensing mechanism using a servo-controlled rotating disc. Each pill slot was aligned with a timing mechanism linked to a Raspberry Pi. The dispensing process was controlled via a real-time clock (RTC) module and was further enhanced with facial recognition for personalized medication delivery. The study illustrated the effectiveness of integrating computer vision and embedded systems in automating healthcare routines. This directly supports the use of a Raspberry Pi and camera in your design, highlighting how time-based, person-specific pill dispensing increases safety and compliance in medication routines.
\newline
\newline
\newline


3. IOP Conference Series Article 
The IOP Conference Series article 10.1088/1757-899X/981/4/042007 contributes to the engineering foundations relevant to hospital robotics by advancing understanding in robotic system integration, sensor use, and control methodologies, typical of research featured in this series. Although full text is not directly available via the provided DOI, IOP Conference Series publications regularly address robot control systems, autonomous navigation techniques, and the application of integrated sensor networks for improved reliability and performance—factors essential to developing effective mobile robots in healthcare contexts. These works reinforce the technical underpinnings necessary for autonomous and semi-autonomous robots to reliably operate in complex indoor environments such as hospitals, enhancing tasks such as logistics transport, environmental interaction, and human-robot collaboration.

