\subsection*{3.1 Components used}
This project consists of both hardware and software components. The hardware components used in this project are discussed below:

\subsubsection*{3.1.1 Johnson motor}
A Johnson motor is a brushed DC motor that contains permanent magnets. It is a high-speed motor with high torque and widely used in robotic applications. In this project, Johnson motors are used for driving the wheels of the robot.

\begin{figure}[H]
    \centering
    \includegraphics[width=0.5\textwidth]{3.1.jpg}
    \caption{Johnson Motor (Source: \cite{7})}
    \label{fig:3.1}
\end{figure}

\subsubsection*{3.1.2 Raspberry Pi 3}
RRaspberry Pi 3 is a credit card-sized single-board computer capable of running an operating system and performing multiple computing tasks. It is equipped with a Broadcom BCM2837, Quad-core Cortex-A53 (ARMv8) 64-bit SoC @ 1.2 GHz, 1 GB RAM, along with USB, HDMI, Ethernet, Wi-Fi, and Bluetooth connectivity.
In this project, the Raspberry Pi 3 is used to  communicate with cloud services and send control signals to the Arduino for further actuation and control operations.

\begin{figure}[H]
    \centering
    \includegraphics[width=0.5\textwidth]{fig3.2.jpg}
    \caption{Raspberry Pi 4 (Source: \cite{8})}
    \label{fig:3.2}
\end{figure}

\subsubsection*{3.1.3 Arduino Mega}
The Arduino Mega 2560 is an open-source microcontroller board based on the Microchip ATmega2560 microcontroller. It features 54 digital I/O pins, 16 analog input pins, 4 hardware serial ports (UARTs), a 16 MHz crystal oscillator, and USB and power jack support.
In this project, the Arduino Mega 2560 is used to control motors and sensors and execute control actions based on the instructions received from the Raspberry Pi.
\begin{figure}[H]
    \centering
    \includegraphics[width=0.5\textwidth]{fig3.3.png}
    \caption{Arduino UNO (Source: \cite{9})}
    \label{fig:3.3}
\end{figure}

\subsubsection*{3.1.4 IR Sensor}
IR (Infrared) sensors are used to detect the path for line-following by emitting and detecting IR light. A group of IR sensors is arranged to detect black lines on a cemented floor. In this project, they help the bot follow the designated path.

\begin{figure}[H]
    \centering
    \includegraphics[width=0.5\textwidth]{3.4.jpg}
    \caption{IR Array Sensor (Source: \cite{10})}
    \label{fig:3.4}
\end{figure}

\subsubsection*{3.1.5 Metal Gear Servo MG996R}
The MG996R is a high-torque servo motor with metal gears and digital control. It provides accurate angle control and is used in the pill dispensing mechanism. It can rotate 0–180 degrees and is known for durability and stability.

\begin{figure}[H]
    \centering
    \includegraphics[width=0.5\textwidth]{3.5.jpg}
    \caption{Metal Gear Servo MG996R (Source: \cite{11})}
    \label{fig:3.5}
\end{figure}
\section{Software}

\textbf{Arduino IDE:} The Arduino Integrated Development Environment (IDE) is used to program the Arduino microcontroller responsible for real-time control of motors, sensors, and actuators. Written in C/C++, the code implements PID algorithms for accurate line following, triggers pill dispensing via servo motors, and reads data from modules such as IR sensors, UHF RFID readers, and ultrasonic sensors. Libraries like \texttt{Servo.h}, \texttt{Wire.h}, and \texttt{SoftwareSerial.h} simplify hardware interactions. Serial communication with the Raspberry Pi enables synchronous operation and system-level decision-making. The IDE’s Serial Monitor is invaluable for debugging sensor readings and validating motor responses, supporting firmware-level coordination for the robot’s motion and dispensing tasks.

\vspace{0.5em}

\textbf{Raspberry Pi \& Camera Module:} The Raspberry Pi 4 Model B, paired with the Raspberry Pi Camera V2 (NoIR), serves as the high-level processing unit of the system. It runs Python scripts for facial recognition using OpenCV and manages scheduled pill dispensing through a timetable system. Sensor feedback and motor status are received from the Arduino via UART serial communication, enabling dynamic coordination based on environmental inputs and patient recognition. The onboard real-time clock (RTC, optional) ensures precise execution of time-based actions. Captured facial data is matched with a patient database to ensure accurate medication delivery. A local SQLite or CSV-based logging system maintains records for real-time auditing and system monitoring.

\vspace{0.5em}

\textbf{SOLIDWORKS:} SOLIDWORKS is a professional computer-aided design (CAD) software widely used for mechanical design and product development. In this project, SOLIDWORKS is used to design and model the physical structure of the robot before fabrication. It enables the creation of accurate 3D models of the robot chassis, compartments, pill-dispensing mechanism, and mounting structures.





%\section{Dataset}
%\lipsum[4-8]