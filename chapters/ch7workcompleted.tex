

This chapter presents the software tools used in the project, the work completed so far, and the remaining tasks planned for future phases of development.

\section{Software Tools Used}

The development of the face-recognition-based medication dispensing system required the integration of multiple software tools and frameworks. The following software components were used:

\begin{itemize}
    \item \textbf{Python}: Used as the primary programming language for implementing the complete system logic due to its simplicity and extensive library support.
    \item \textbf{OpenCV}: Used for real-time video capture, face detection, image preprocessing, and visual feedback.
    \item \textbf{Face Recognition / Deep Learning Library}: Used for extracting facial embeddings and performing identity matching using cosine similarity.
    \item \textbf{SQLite Database}: Used to store patient information, facial embeddings, medication schedules, and audit logs.
    \item \textbf{NumPy}: Used for numerical computations and vector similarity calculations.
    \item \textbf{Raspberry Pi OS (Planned)}: Intended operating system for deployment on embedded hardware.
\end{itemize}

These tools collectively ensure real-time performance, reliability, and scalability of the system.
\newpage
\section{Work Completed So Far}

The core functionality of the proposed medication dispensing system has been successfully implemented in simulation mode. The completed work is described in detail below.

\subsection{Face Recognition System Implementation}

A complete face recognition pipeline has been implemented using computer vision and deep learning techniques.

\begin{itemize}
    \item Face detection and facial embedding extraction have been implemented.
    \item Multiple face embeddings per patient are stored to improve recognition accuracy.
    \item Cosine similarity is used to compare live face data with stored embeddings.
    \item Confidence thresholds are applied to classify recognition results as recognized, uncertain, or unknown.
\end{itemize}

This approach minimizes false recognition and ensures safe system behavior.

\subsection{Patient Registration Module}

A patient registration system has been developed to manage user enrollment.

\begin{itemize}
    \item New patients can be registered, and additional face samples can be added for existing patients.
    \item Multiple face samples are captured under different conditions.
    \item The system enforces single-face capture during registration.
    \item All patient data and embeddings are stored in a structured SQLite database.
\end{itemize}

This module ensures accurate and reliable face identification.

\subsection{Medication Scheduling System}

A medication scheduling subsystem has been implemented to control medication access.

\begin{itemize}
    \item Medication schedules include patient ID, compartment number, and time window.
    \item The system verifies whether the current time falls within the scheduled window.
    \item The scheduler returns states such as DISPENSE, NOT\_TIME, ALREADY\_TAKEN, or ERROR.
\end{itemize}

This ensures medications are dispensed strictly according to prescription timings.

\subsection{Safety-Critical Decision Logic}

Several safety mechanisms have been integrated to prevent misuse.

\begin{itemize}
    \item Confirmation delay requiring the patient to remain in front of the camera.
    \item Cooldown period to prevent repeated dispensing.
    \item Blocking of dispensing when multiple faces are detected.
    \item Prevention of dispensing to unknown or unregistered individuals.
    \item Graceful handling of camera and database errors.
\end{itemize}

These measures make the system suitable for healthcare applications.

\subsection{Real-Time Performance Optimization}

The system has been optimized for real-time operation.

\begin{itemize}
    \item Frame skipping is implemented to reduce computational load.
    \item Real-time FPS monitoring is displayed.
    \item Stable bounding box and label display prevents flickering.
    \item Recognition performance remains consistent under movement.
\end{itemize}

This prepares the system for deployment on resource-constrained devices.

\subsection{Event-Based Audit Logging System}

An audit logging mechanism has been implemented for accountability.

\begin{itemize}
    \item Only system state changes are logged.
    \item Logs include timestamp, patient identity, action taken, confidence score, and compartment number.
    \item Duplicate log entries are avoided using state tracking.
    \item Logs are stored in an SQLite database.
\end{itemize}

This is essential for monitoring and medical auditing.

\subsection{Modular System Architecture}

The system follows a modular design approach.

\begin{itemize}
    \item Separate modules for recognition, scheduling, dispensing, logging, and database handling.
    \item Easy hardware replacement and feature extension.
    \item Designed for future Raspberry Pi deployment.
\end{itemize}

\section{Work Remaining / Future Work}

Although the core system is functional in simulation, the following tasks remain.

\subsection{Raspberry Pi Deployment}

\begin{itemize}
    \item Port the system to Raspberry Pi.
    \item Integrate Raspberry Pi Camera module.
    \item Optimize performance for embedded hardware.
\end{itemize}

\subsection{Servo Motor Integration}

\begin{itemize}
    \item Integrate servo motors using GPIO pins.
    \item Map medication compartments to physical dispensing units.
    \item Implement calibration and fail-safe mechanisms.
\end{itemize}

\subsection{Dashboard and Monitoring Interface}

\begin{itemize}
    \item Develop a caregiver dashboard.
    \item Display patient schedules and dispensing history.
    \item Enable alert and warning notifications.
\end{itemize}

\subsection{Additional Enhancements}

\begin{itemize}
    \item Voice alerts for patients.
    \item Emergency notifications for caregivers.
    \item Multi-factor authentication using face and voice.
\end{itemize}

\subsection{Testing and Validation}

\begin{itemize}
    \item Long-term real-world testing.
    \item User acceptance testing.
    \item Stress testing with multiple users and schedules.
\end{itemize}

\section{Summary}

The project has successfully achieved its primary objectives, including accurate face recognition, safe medication scheduling, robust decision-making, and real-time performance. Future work will focus on hardware integration, user interface development, and deployment optimization.

