\section{Background Theory}
The rapid integration of robotics into healthcare environments is transforming traditional caregiving approaches. With rising demands for efficient patient management in hospitals, hostels, and elderly care centers, autonomous service robots are becoming an innovative solution. These robots, especially those capable of delivering medications and supplies, offer consistent, scheduled, and contactless services that reduce human error and free up valuable time for medical personnel.
Among the various robotic systems being developed, line-following robots have gained attention for their simplicity, reliability, and adaptability in structured indoor environments like hospital wards or dormitories with fixed bed layouts. These robots navigate by following predefined paths, typically using infrared (IR) sensors to detect contrasting lines on the floor. Such navigation is especially useful for environments where precise, repeatable routes are needed to deliver medication trays to specific beds.
The proposed system enhances this concept by introducing a three-compartment modular robot. The bottom compartment houses motion components, including motors, wheels, sensors, and batteries, making it the foundation for navigation and mobility that follows the line following mechanism. The middle compartment is dedicated to general storage, such as medical equipment, water, or additional medicine trays. The top compartment features a smart pill dispenser powered by a Raspberry Pi and a camera. This setup allows the robot to identify patients using facial recognition algorithms and dispense medicine according to a preset time schedule, minimizing the chances of human error and increasing medication adherence.
To ensure robust interaction and real-time tracking, the system can integrate wireless communication modules (e.g., Wi-Fi or Bluetooth) to connect with a central server or a mobile application. This allows healthcare providers to monitor the robot’s location, view pill dispensing logs, and modify schedules as needed. The use of AI in facial recognition and scheduling also paves the way for personalized patient care, enabling the robot to adapt to individual needs.
Modularity in the robot’s design adds another layer of practicality. Each compartment is detachable and independently upgradable, enabling easy maintenance, swift replacements, and the possibility of repurposing the robot for different tasks (e.g., food delivery, waste collection) by simply swapping modules. Lightweight yet durable materials like 3D-printed PLA or ABS plastic are ideal for building these components, ensuring comfort, stability, and structural integrity.
In the broader context, this project reflects the growing intersection of robotics, AI, and healthcare a convergence that is expected to redefine service delivery in hospitals. By aligning routine medical tasks with intelligent automation, such systems not only improve efficiency but also offer a glimpse into the future of compassionate, technology-driven patient care.



\vspace{1.5\baselineskip} 

\section{Problem Statement}
Modern healthcare systems particularly in resource constrained settings such as public hospitals, mass hostels, and elderly care facilities face ongoing challenges in ensuring timely, consistent, and personalized delivery of medication. Relying on human staff for routine tasks such as distributing pills or tracking medication schedules often results in inefficiencies, human error, and missed doses, especially during peak workloads or emergencies. While existing medical robots offer partial automation, they are typically expensive, non-modular, or dependent on proprietary software, limiting their adaptability and scalability in practical environments.
Furthermore, most delivery bots lack intelligent interaction capabilities. They often depend solely on line-following navigation without integrating patient identification, flexible routing, or automated scheduling, making them unsuitable for truly personalized or responsive care. Pill-dispensing mechanisms in these systems are also generally basic, lacking integration with facial recognition, scheduling, or patient feedback mechanisms.
This project addresses these limitations by developing a low-cost, modular, and intelligent line-following medical assistant robot equipped with a smart pill dispensing system. The robot combines real-time navigation, multi-compartment storage, and a top-mounted AI-driven module featuring a Raspberry Pi and camera for facial recognition and time-based pill dispensing. The modular design allows each compartment—motion, storage, and dispenser—to be independently upgraded or replaced, ensuring long-term flexibility, easy maintenance, and expandability.
By integrating low-cost hardware components with artificial intelligence and wireless communication, this project aims to deliver a scalable solution that enhances healthcare automation. It enables accurate, personalized medication delivery while reducing staff workload, minimizing human error, and supporting future integration with mobile apps or hospital information systems. Ultimately, the system reimagines patient-care interaction by aligning routine medical services with smart, autonomous, and context-aware robotic assistance.
 

\vspace{1.5\baselineskip} 

\section{Objectives}
The primary objective of this project is to design and develop a modular, intelligent line-following medical assistant robot capable of navigating hospital or hostel rooms and delivering personalized medications through an AI-powered pill dispenser. The key objectives are:
\begin{itemize}
    
    \item To develop a robust line-following mechanism using infrared sensors for reliable indoor navigation.
    \item To implement a modular three-compartment architecture consisting of:
    \subitem Bottom: Motion and power unit
    \subitem Middle: Storage trays
    \subitem Top: Raspberry Pi-powered smart pill dispenser
    \item To integrate AI-based facial recognition for identifying patients and ensuring correct pill delivery.
    \item To schedule pill dispensing based on real-time clock or external input for personalized and timely delivery.
    \item To create a centralized mobile or web-based application for monitoring robot logs, modifying pill schedules, and tracking delivery status.
    \item To maintain affordability, scalability, and ease of maintenance by using low-cost sensors, 3D-printed housings, and open-source software/hardware.
\end{itemize}

\vspace{1.5\baselineskip} 

\section{Scope}
This project is focused on creating a prototype of a healthcare delivery robot optimized for structured environments like hospital wards and student hostels. The scope includes:
\begin{itemize}
    \item Development of an autonomous line-following base that can navigate pre-defined paths.
    \item Designing 3D-printed modular compartments with quick-connect mechanisms for easy upgrades or replacements.
    \item Implementing facial recognition using a Raspberry Pi and camera to authenticate patients before dispensing medication.
    \item Building a basic pill dispensing mechanism controlled by a servo or motor, programmed via scheduled inputs.
    \item Building a basic pill dispensing mechanism controlled by a servo or motor, programmed via scheduled inputs.
    \item Data logging for dosage history, delivery time, and patient interaction.
\end{itemize}
\vspace{1.5\baselineskip}
\section{Applications}

\begin{itemize}
    \item \textbf{Healthcare Automation} \\
    Automates pill delivery and dispensing in hospitals, elderly care centers, and quarantine wards, reducing manual errors and ensuring timely medication administration.

    \item \textbf{Assistive Technology} \\
    Supports patients with disabilities or the elderly by providing timely and accurate medication without needing constant supervision from healthcare staff.

    \item \textbf{Smart Environments / IoT Applications} \\
    Integrates with hospital management systems for real-time tracking of medication schedules, patient recognition, and route optimization using IoT connectivity.

    \item \textbf{Robotics Control} \\
    Demonstrates practical use of autonomous navigation, sensor integration, and servo-controlled dispensing mechanisms in robotics.

    \item \textbf{Educational Tools} \\
    Serves as a multidisciplinary STEM project combining electronics, embedded systems, AI (for face recognition), and mechanical design, ideal for learning and teaching.

    \item \textbf{Pharmaceutical Logistics} \\
    Assists in internal logistics within medical facilities by distributing not only pills but also medical supplies or documents efficiently and contact-free.
\end{itemize}
