\section{System Block Diagram}

\begin{figure}[h!]
    \centering
    \begin{tikzpicture}[node distance=2cm, auto, >=latex', thick]
        % Sensor nodes
        \node[draw, rectangle, rounded corners, minimum width=2cm, minimum height=1cm] (mpu) {MPU6050};
        \node[draw, rectangle, rounded corners, minimum width=2cm, minimum height=1cm, below left=1cm and 2cm of mpu] (flex) {Flex Sensors};
        \node[draw, rectangle, rounded corners, minimum width=2cm, minimum height=1cm, below right=1cm and 2cm of mpu] (tactile) {Tactile Switches};
        
        % Processing node
        \node[draw, rectangle, minimum width=3cm, minimum height=1cm, below=3cm of mpu] (micro) {Arduino Mega};
        
        % Calibration node
        \node[draw, rectangle, minimum width=2.5cm, minimum height=0.8cm, below=1.5cm of micro] (calibration) {Data Calibration};
        
        % Communication node
        \node[draw, rectangle, minimum width=3cm, minimum height=1cm, below=1.5cm of calibration] (comm) {Serial Communication};
        
        % Software nodes
        \node[draw, rectangle, minimum width=3cm, minimum height=1cm, below=1.5cm of comm] (pc) {PC Interface};
        \node[draw, rectangle, minimum width=3cm, minimum height=1cm, below=1.5cm of pc] (unreal) {Unreal Engine};
        
        % Output node
        \node[draw, rectangle, minimum width=3.5cm, minimum height=1cm, below=1.5cm of unreal] (virtual) {3D Hand Model};

        % Connections
        \draw[->] (mpu) -- (micro);
        \draw[->] (flex) -- (micro);
        \draw[->] (tactile) -- (micro);
        \draw[->] (micro) -- (calibration);
        \draw[->] (calibration) -- (comm);
        \draw[->] (comm) -- (pc);
        \draw[->] (pc) -- (unreal);
        \draw[->] (unreal) -- (virtual);

    \end{tikzpicture}
    \caption{Detailed Block Diagram of Hand Motion Replicator System}
    \label{fig:block_diagram}
\end{figure}

\section{Algorithm}
\begin{enumerate}
    \item System Initialization
    \begin{itemize}
        \item Configure Raspberry Pi GPIOs for motor control
        \item Initialize camera module and load known patient face encodings
        \item Set up serial communication between Raspberry Pi and Arduino
        \item Initialize Arduino for RFID, IR line sensors, and obstacle detection
        \item Configure RTC module for pill dispensing schedule
    \end{itemize}

    \item Navigation and Obstacle Handling
    \begin{itemize}
        \item Read line sensor data for path following
        \item Use PID control to adjust wheel speeds for stable line tracking
        \item Monitor ultrasonic or IR sensors for obstacle detection
        \item Identify patient delivery zones using floor markers or RFID
    \end{itemize}

    \item Patient Identification
    \begin{itemize}
        \item Capture image using Pi Camera at each stop
        \item Perform face recognition using pre-stored encodings
        \item Confirm patient identity before dispensing
    \end{itemize}

    \item Pill Dispensing
    \begin{itemize}
        \item Verify time with RTC or cloud schedule
        \item Match medicine to patient's current prescription
        \item Activate servo/stepper mechanism to release pills
        \item Log successful or failed delivery
    \end{itemize}

    \item Data Logging and Communication
    \begin{itemize}
        \item Log patient ID, timestamp, and medicine dispensed
        \item Send updates to local storage or cloud database
        \item Support real-time monitoring via a dashboard
    \end{itemize}

    \item Return and Remote Update
    \begin{itemize}
        \item Navigate to next patient or return to base station
        \item Sync prescription or face data remotely via web interface
        \item Allow admin to register new patients and schedules
    \end{itemize}
\end{enumerate}




