\section{Flowcharts}

\subsection{Overall System Workflow}

\tikzstyle{startstop} = [rectangle, rounded corners, minimum width=3cm, minimum height=1cm, text centered, draw=black, fill=white]
\tikzstyle{process} = [rectangle, minimum width=3.2cm, minimum height=1cm, text centered, draw=black, fill=white]
\tikzstyle{decision} = [diamond, aspect=2, minimum width=3.5cm, minimum height=1cm, text centered, draw=black, fill=white]
\tikzstyle{arrow} = [thick,->,>=stealth]

\begin{figure}[H]
\centering
\resizebox{\textwidth}{!}{ % scales to page width
\begin{tikzpicture}[node distance=1.5cm and 2.5cm]

% Main vertical nodes
\node (start) [startstop] {System Start};
\node (init) [process, below of=start] {Initialize RPi \& Arduino};
\node (nav) [process, below of=init] {Start Line Following};
\node (zone) [decision, below of=nav, yshift=-0.5cm] {Reached Patient Zone?};

% Yes branch (right)
\node (capture) [process, right=4cm of zone] {Capture Face Image};
\node (match) [process, below of=capture] {Facial Recognition};
\node (verify) [process, below of=match] {Verify Schedule (RTC)};
\node (dispense) [process, below of=verify] {Dispense Medication};
\node (log) [process, below of=dispense] {Log Data \& Notify};
\node (continue) [process, below of=log] {Continue/Return to Base};
\node (end) [startstop, below of=continue] {System End};

% No branch (left)
\node (move) [process, left=4cm of zone] {Move to Another Patient};

% Arrows - main flow
\draw [arrow] (start) -- (init);
\draw [arrow] (init) -- (nav);
\draw [arrow] (nav) -- (zone);

% Yes path
\draw [arrow] (zone.east) -- node[above] {Yes} (capture.west);
\draw [arrow] (capture) -- (match);
\draw [arrow] (match) -- (verify);
\draw [arrow] (verify) -- (dispense);
\draw [arrow] (dispense) -- (log);
\draw [arrow] (log) -- (continue);
\draw [arrow] (continue) -- (end);

% No path
\draw [arrow] (zone.west) -- node[above] {No} (move.east);
\draw [arrow] (move) |- (nav.west);

\end{tikzpicture}
}
\caption{Overall System Workflow}
\label{fig:overall_system_workflow}
\end{figure}


\subsection{Pill Dispensing Subsystem}

\begin{figure}[H]
\centering
\begin{tikzpicture}[node distance=1.5cm and 3cm]

\node (start) [startstop] {At Patient Zone};
\node (face) [process, below of=start] {Face Recognition};
\node (match) [decision, below of=face, yshift=-0.5cm] {Face Recognized?};
\node (check) [process, right of=match, xshift=5cm] {Check Schedule};
\node (abort) [process, left of=match, xshift=-5cm] {Abort \& Log};
\node (endAbort) [startstop, below of=abort] {Subsystem End};
\node (dispense) [process, below of=check] {Activate Servo \& Dispense};
\node (log) [process, below of=dispense] {Log Success};
\node (end) [startstop, below of=log] {Subsystem End};

\draw [arrow] (start) -- (face);
\draw [arrow] (face) -- (match);
\draw [arrow] (match.east) -- (check.west) node[midway, above] {Yes};
\draw [arrow] (match.west) -- (abort.east) node[midway, above] {No};
\draw [arrow] (abort) -- (endAbort);
\draw [arrow] (check) -- (dispense);
\draw [arrow] (dispense) -- (log);
\draw [arrow] (log) -- (end);

\end{tikzpicture}
\caption{Pill Dispensing Subsystem Flowchart}
\label{fig:pill_dispensing_subsystem}
\end{figure}


\subsection{Navigation and Obstacle Handling}

\begin{figure}[H]
\centering
\begin{tikzpicture}[node distance=1.5cm and 3cm]

\node (start) [startstop] {Start Line Following};
\node (read) [process, below of=start] {Read IR Sensor Data};
\node (pid) [process, below of=read] {PID-based Motor Control};
\node (obstacle) [decision, below of=pid, yshift=-0.5cm] {Obstacle Detected?};
\node (stop) [process, left of=obstacle, xshift=-5cm] {Stop \& Wait};
\node (endStop) [startstop, below of=stop] {Navigation End};
\node (move) [process, right of=obstacle, xshift=5cm] {Continue to Next Zone};
\node (endMove) [startstop, below of=move] {Navigation End};

\draw [arrow] (start) -- (read);
\draw [arrow] (read) -- (pid);
\draw [arrow] (pid) -- (obstacle);
\draw [arrow] (obstacle.west) -- (stop.east) node[midway, above] {Yes};
\draw [arrow] (obstacle.east) -- (move.west) node[midway, above] {No};
\draw [arrow] (stop) -- (endStop);
\draw [arrow] (move) -- (endMove);

\end{tikzpicture}
\caption{Navigation and Obstacle Handling Flowchart}
\label{fig:navigation_obstacle_handling}
\end{figure}